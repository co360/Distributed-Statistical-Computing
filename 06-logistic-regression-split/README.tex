\section{分块Logistics回归}\label{ux5b9eux4f8bux5206ux6790ux57faux4e8emapreduceux7684logisticsux56deux5f52}

\subsection{准备知识}\label{ux51c6ux5907ux77e5ux8bc6}

\begin{itemize}
\itemsep1pt\parskip0pt\parsep0pt
\item
  Logistics回归
\item
  R和Python
\item
  Hadoop Streaming
\end{itemize}

\subsection{研究思路}\label{ux7814ux7a76ux601dux8def}

Logit模型(Logistic model),是最早的离散选择模型。Logit模型的求解速度快、应用方便,可以对预
测的结果进行比较和检验,在社会学、生物统计学、临床、数量心理学、计量经济学、市场营销等统计
实证分析中广为应用。本文以R软件中的Orange Juice数据为研究对象,通过描述性分析对消费者橙汁品
牌选择的偏好进行初步分析,继而分别建立了Logit回归模型,对消费数据进行拟合分析。最后
在MapReduce实践部分,展示了完整的从数据获取到Map函数、Reduce函数以及输出结果的过程。在服务
器上使用MapRuducer的方法,对Logistics回归进行并行计算操作。MapReduce的具体流程如下:

\begin{enumerate}
\def\labelenumi{\arabic{enumi}.}
\item
  首先将数据分成k个不同的块。
\item
  对每一块的数据进行logistic回归估计出自变量的系数;Reduce函数则对每一块的自变量系数进行
  平均,作为最终的结果。
\end{enumerate}

可以证明,当数据量n和每一个子块中的数据量m足够大的情况下,用这种MapReduce方法计算出来的系
数估计与真实值具有一致性。

\subsection{数据准备}\label{ux6570ux636eux51c6ux5907}

本研究所用样本数据均来自R软件ISLR包中的OJ数据,样本中包含1070个消费者购买Citrus
Hill 牌或 Minute Maid牌橙汁的行为数据。1070个样本数据的18个变量包
含Purchase、Store7、StoreID和Store4个分类变量,和其余的14个定量变量,用于以下研究分析。

\subsection{建立Logit回归模型}\label{ux5efaux7acblogitux56deux5f52ux6a21ux578b}

将购买(Purchase)作为本文的因变量,将购买``CH''定义为0,``购买MM''定义为1,选取CH售价
(PriceCH),MM售价(PriceMM),CH折扣(DiscCH),MM折扣(DiscMM),CH特殊指标
(SpecialCH),MM特殊指标(SpecialMM),CH忠诚度(LoyalCH)几个指标作为自变量建立Logit回归
模型。MapReduce中首先将数据打乱,随后分成不同的3块,以下分别给出3块回归的结果和最后Reduce后
的结果,并和利用全部数据整体进行对比。

\subsubsection{Mapper部分}\label{mapperux90e8ux5206}

\begin{lstlisting}
	#! usr/bin/env python
	import sys
	import pandas as pd
	from statsmodels.api import *
	import numpy as np

	Purchase = []
	WeekofPurchase = []
	StoreID = []
	PriceCH = []
	PriceMM = []
	DiscCH = []
	DiscMM = []
	SpecialCH = []
	SpecialMM = []
	LoyalCH = []
	Store7 = []
	PctDiscMM =[]
	PctDiscMM = []
	Store = []

	#打开文件
	f = sys.stdin
	#读取数据
	for line in f.readlines():
	    vrb = line.split(',')
	    Purchase.append(vrb[1])
	    WeekofPurchase.append(float(vrb[2]))
	    StoreID.append(float(vrb[3]))
	    PriceCH.append(float(vrb[4]))
	    PriceMM.append(float(vrb[5]))
	    DiscCH.append(float(vrb[6]))
	    DiscMM.append(float(vrb[7]))
	    SpecialCH.append(float(vrb[8]))
	    SpecialMM.append(float(vrb[9]))
	    LoyalCH.append(float(vrb[10]))
	    Store7.append(vrb[14])
	    PctDiscMM.append(float(vrb[15]))
	    PctDiscMM.append(float(vrb[16]))
	    Store.append(float(vrb[18]))

	f.close

	#因变量选择Purchase中的MM,如果是MM则取1否则为0
	MM = pd.get_dummies(Purchase)#将分类型变量转化为0,1
	MM = MM[MM.columns[1]]#选择MM为真的一列

	#选取自变量
	xdata = pd.DataFrame([PriceCH,PriceMM,DiscCH,DiscMM,SpecialCH,SpecialMM,LoyalCH])
	xdata=xdata.T
	xdata.columns=['PriceCH','PriceMM','DiscCH','DiscMM','SpecialCH','SpecialMM','LoyalCH']
	xdata['intercept']=1.0

	#打乱顺序
	shuffle = np.random.permutation(1069)
	MM = MM[shuffle]
	xdata = xdata.ix[shuffle]

	l = len(MM)
	n=3#划分不同的几块,这里令n=3
	step = l/n

	#对前n-1块进行计算
	for i in range(n-1):
	    logitmodel = Logit(MM[step*i:step*(i+1)],xdata[step*i:step*(i+1)])
	    res = logitmodel.fit(disp=0)
	    for j in range(len(res.params)):
	        print res.params [j]
	    print ','

	#对最后一块进行计算
	logitmodel = Logit(MM[step*(n-1):l],xdata[step*(n-1):l])
	res = logitmodel.fit(disp=0)
	for j in range(len(res.params)):
	    print res.params [j]
\end{lstlisting}

\subsubsection{Reducer部分}\label{reducerux90e8ux5206}

\begin{lstlisting}
	import sys
	import pandas as pd
	#读取数据
	f = sys.stdin
	fields = f.read()
	#整理数据
	fields = fields.split(',')#每一块数据的结果是按“,”分割的,重新用“,”划分成列表
	data = [field.split('\n') for field in fields]#每一块中自变量的系数
	for i in range(len(data)):
	    while '' in data[i]:#去除列表中的空格
	        data[i].remove('')
	data = pd.DataFrame(data)#重新整理成数据框
	data.columns=['PriceCH','PriceMM','DiscCH','DiscMM','SpecialCH','SpecialMM','LoyalCH','Intercept']
	data = data.astype(float)#将数据框中的元素类型转化为float
	print data
	print data.mean()
\end{lstlisting}

\subsubsection{实践结论}\label{ux5b9eux8df5ux7ed3ux8bba}

观察Logit全模型的估计结果,可以发现CH标价(PriceCH)、MM标价(PriceMM)、MM折扣
(DiscMM)、CH品牌忠诚度(LoyalCH)、MM折扣比例(PctDiscMM)对Purchase有显著的影响。一定程
度上对Purchase有解释作用,而其他变量对因变量解释性并不显著。通过对比MapReduce的结果和利用全
部数据同时建模的结果可以发现,MapReduce的结果精确度还是比较高的,为了避免是一次实验造成的随
机性结果,又重复多次进行计算,发现大多数变量的系数依然较为只有``CH特殊指标(SpecialCH)``这
一个变量的系数有可能出现符号相反的结果,考虑到其系数本身就接近于0,同时并不显著,所以不能改
变我们对最终结果比较精确的判断。在对三个子块进行建模得到的系数进行比较后,发现不同子块之间
的个别变量的系数差别还是比较大的,例如``CH价格(PriceCH)''三次得到结果分别
是6.28,1.44和4.39,这可能与数据的分布有关系,但是不影响最终Reduce的效果。
