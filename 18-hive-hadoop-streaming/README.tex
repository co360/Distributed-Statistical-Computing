\section{Hive实例:Hive Streaming 交互计算}\label{ux5b9eux4f8bux5206ux6790hadoop-streaming-ux4e0e-hive-ux4ea4ux4e92ux8ba1ux7b97ux5b9eux4f8b}

\subsection{准备知识}\label{ux51c6ux5907ux77e5ux8bc6}

\begin{itemize}
\itemsep1pt\parskip0pt\parsep0pt
\item
  Hive
\item
  Hadoop Streaming
\item
  Python
\end{itemize}

\subsection{数据准备}\label{ux6570ux636eux51c6ux5907}

与Hadoop类似,Hive也支持使用JAVA之外的语言编写Streaming程序扩展Hive,实现Hive中的函数无法实
现的功能。开发Hive的Streaming程序和开发Hadoop的Streaming程序是相同的,都是从标准输入中读取
数据,将结果写出到标准输出中。Hive中提供多个语法来使用Streaming,包括\texttt{map()},
\texttt{reduce()}, \texttt{TRANSFORM( )},使用\texttt{TRANSFROM()}的情形较多。例如,我们现
在需要将表格中的日期字符串转换成季节字符串形式,若现在的日期形式为YYYY-MM-DD,通过变换,希
望输出YYYYQ1的季节字符串形式。我们在Hive中创建表格birdata,并载入。测试数据:

\begin{lstlisting}
AAA    1990-02-01
BBB    1984-01-16
CCC    1988-07-01
\end{lstlisting}

下面演示使用Hive Streaming进行字符串的转换。
首先写Python脚本进行数据的转换:

\begin{lstlisting}
#!/usr/bin/env python
#-*- coding:utf-8 -*-

import sys
#转换函数
def get_date_year(pdate):
    (year_val,month_val) = (pdate[0:4],pdate[4:6])
    quarter_index = (int(month_val)-1)/3+1
    quarter_str = "%sQ%d" % (year_val,quarter_index)
    return quarter_str

for line in sys.stdin:
    line = str(line).strip()
    if not line: continue
    fields = line.split("\t")
    date_val = str(fields[-1]).replace("-","")
    output_fields = fields[:-1]+[get_date_year(date_val)]
    print "\t".join(output_fields)
\end{lstlisting}

再次,在Hive中调用上面的Streaming脚本;这个调用中包含两个步骤,第一步需要以绝对路径的方式添加脚本;第二步需要用\texttt{TRANSFORM\ldots{}USING\ldots{}AS}句式进行调用。
\texttt{use test}; \texttt{add file /Users/hive/quarter.py}; \texttt{select TRANSFROM(birthday)
using `quarter.py' as (bir\_quarter) from birdata};
注意,使用这种方式调用Streaming,\texttt{select}中除了\\\texttt{TRANSPORM}外不能有别的字段。在大数据集上使用Streaming时,一般都需要和\texttt{DISTRIBUTE
BY\ldots{}}, \texttt{SORT BY}句式一起使用,解决单节点负载过重问题。
